% Options for packages loaded elsewhere
\PassOptionsToPackage{unicode}{hyperref}
\PassOptionsToPackage{hyphens}{url}
\PassOptionsToPackage{dvipsnames,svgnames,x11names}{xcolor}
%
\documentclass[
  letterpaper,
  DIV=11,
  numbers=noendperiod]{scrartcl}

\usepackage{amsmath,amssymb}
\usepackage{iftex}
\ifPDFTeX
  \usepackage[T1]{fontenc}
  \usepackage[utf8]{inputenc}
  \usepackage{textcomp} % provide euro and other symbols
\else % if luatex or xetex
  \usepackage{unicode-math}
  \defaultfontfeatures{Scale=MatchLowercase}
  \defaultfontfeatures[\rmfamily]{Ligatures=TeX,Scale=1}
\fi
\usepackage{lmodern}
\ifPDFTeX\else  
    % xetex/luatex font selection
\fi
% Use upquote if available, for straight quotes in verbatim environments
\IfFileExists{upquote.sty}{\usepackage{upquote}}{}
\IfFileExists{microtype.sty}{% use microtype if available
  \usepackage[]{microtype}
  \UseMicrotypeSet[protrusion]{basicmath} % disable protrusion for tt fonts
}{}
\makeatletter
\@ifundefined{KOMAClassName}{% if non-KOMA class
  \IfFileExists{parskip.sty}{%
    \usepackage{parskip}
  }{% else
    \setlength{\parindent}{0pt}
    \setlength{\parskip}{6pt plus 2pt minus 1pt}}
}{% if KOMA class
  \KOMAoptions{parskip=half}}
\makeatother
\usepackage{xcolor}
\setlength{\emergencystretch}{3em} % prevent overfull lines
\setcounter{secnumdepth}{-\maxdimen} % remove section numbering
% Make \paragraph and \subparagraph free-standing
\makeatletter
\ifx\paragraph\undefined\else
  \let\oldparagraph\paragraph
  \renewcommand{\paragraph}{
    \@ifstar
      \xxxParagraphStar
      \xxxParagraphNoStar
  }
  \newcommand{\xxxParagraphStar}[1]{\oldparagraph*{#1}\mbox{}}
  \newcommand{\xxxParagraphNoStar}[1]{\oldparagraph{#1}\mbox{}}
\fi
\ifx\subparagraph\undefined\else
  \let\oldsubparagraph\subparagraph
  \renewcommand{\subparagraph}{
    \@ifstar
      \xxxSubParagraphStar
      \xxxSubParagraphNoStar
  }
  \newcommand{\xxxSubParagraphStar}[1]{\oldsubparagraph*{#1}\mbox{}}
  \newcommand{\xxxSubParagraphNoStar}[1]{\oldsubparagraph{#1}\mbox{}}
\fi
\makeatother


\providecommand{\tightlist}{%
  \setlength{\itemsep}{0pt}\setlength{\parskip}{0pt}}\usepackage{longtable,booktabs,array}
\usepackage{calc} % for calculating minipage widths
% Correct order of tables after \paragraph or \subparagraph
\usepackage{etoolbox}
\makeatletter
\patchcmd\longtable{\par}{\if@noskipsec\mbox{}\fi\par}{}{}
\makeatother
% Allow footnotes in longtable head/foot
\IfFileExists{footnotehyper.sty}{\usepackage{footnotehyper}}{\usepackage{footnote}}
\makesavenoteenv{longtable}
\usepackage{graphicx}
\makeatletter
\def\maxwidth{\ifdim\Gin@nat@width>\linewidth\linewidth\else\Gin@nat@width\fi}
\def\maxheight{\ifdim\Gin@nat@height>\textheight\textheight\else\Gin@nat@height\fi}
\makeatother
% Scale images if necessary, so that they will not overflow the page
% margins by default, and it is still possible to overwrite the defaults
% using explicit options in \includegraphics[width, height, ...]{}
\setkeys{Gin}{width=\maxwidth,height=\maxheight,keepaspectratio}
% Set default figure placement to htbp
\makeatletter
\def\fps@figure{htbp}
\makeatother

\KOMAoption{captions}{tableheading}
\makeatletter
\@ifpackageloaded{caption}{}{\usepackage{caption}}
\AtBeginDocument{%
\ifdefined\contentsname
  \renewcommand*\contentsname{Table of contents}
\else
  \newcommand\contentsname{Table of contents}
\fi
\ifdefined\listfigurename
  \renewcommand*\listfigurename{List of Figures}
\else
  \newcommand\listfigurename{List of Figures}
\fi
\ifdefined\listtablename
  \renewcommand*\listtablename{List of Tables}
\else
  \newcommand\listtablename{List of Tables}
\fi
\ifdefined\figurename
  \renewcommand*\figurename{Figure}
\else
  \newcommand\figurename{Figure}
\fi
\ifdefined\tablename
  \renewcommand*\tablename{Table}
\else
  \newcommand\tablename{Table}
\fi
}
\@ifpackageloaded{float}{}{\usepackage{float}}
\floatstyle{ruled}
\@ifundefined{c@chapter}{\newfloat{codelisting}{h}{lop}}{\newfloat{codelisting}{h}{lop}[chapter]}
\floatname{codelisting}{Listing}
\newcommand*\listoflistings{\listof{codelisting}{List of Listings}}
\makeatother
\makeatletter
\makeatother
\makeatletter
\@ifpackageloaded{caption}{}{\usepackage{caption}}
\@ifpackageloaded{subcaption}{}{\usepackage{subcaption}}
\makeatother

\ifLuaTeX
  \usepackage{selnolig}  % disable illegal ligatures
\fi
\usepackage{bookmark}

\IfFileExists{xurl.sty}{\usepackage{xurl}}{} % add URL line breaks if available
\urlstyle{same} % disable monospaced font for URLs
\hypersetup{
  pdftitle={LABORATORIO N° 2 -- ADQUISICIÓN DE SEÑALES Y GRAFICACIÓN EN ARDUINO},
  colorlinks=true,
  linkcolor={blue},
  filecolor={Maroon},
  citecolor={Blue},
  urlcolor={Blue},
  pdfcreator={LaTeX via pandoc}}


\title{LABORATORIO N° 2 -- ADQUISICIÓN DE SEÑALES Y GRAFICACIÓN EN
ARDUINO}
\author{}
\date{}

\begin{document}
\maketitle


\subsection{Tabla de contenidos:}\label{tabla-de-contenidos}

\begin{enumerate}
\def\labelenumi{\arabic{enumi}.}
\tightlist
\item
  \hyperref[1-objetivos-especuxedficos-de-la-pruxe1ctica]{Objetivos
  específicos de la práctica}
\item
  \hyperref[2-materiales-y-equipo]{Materiales y equipo}
\item
  \hyperref[3-procedimiento]{Procedimiento}
\item
  \hyperref[4-entregable]{Entregable}
\end{enumerate}

\subsection{1. Objetivos específicos de la
práctica}\label{objetivos-especuxedficos-de-la-pruxe1ctica}

\begin{itemize}
\tightlist
\item
  Adquirir señales conocidas como señal cuadrada, triangular, senoidal,
  rampa, etc.
\item
  Entender los criterios de selección de la frecuencia de muestreo.
\item
  Manipular y configurar adecuadamente una fuente de alimentación
  regulable; multímetro digital; Generador de señales y osciloscopio
  digital.
\end{itemize}

\subsection{2. Materiales y equipo}\label{materiales-y-equipo}

\begin{longtable}[]{@{}
  >{\raggedright\arraybackslash}p{(\columnwidth - 2\tabcolsep) * \real{0.5135}}
  >{\raggedright\arraybackslash}p{(\columnwidth - 2\tabcolsep) * \real{0.4865}}@{}}
\toprule\noalign{}
\begin{minipage}[b]{\linewidth}\raggedright
Equipo
\end{minipage} & \begin{minipage}[b]{\linewidth}\raggedright
Materiales
\end{minipage} \\
\midrule\noalign{}
\endhead
\bottomrule\noalign{}
\endlastfoot
\textbf{Cantidad} & \textbf{Modelo} \\
1 & AFG1022 \\
1 & TBS 1000C Series \\
- & Cable BNC Male-Male \\
- & Punta de osciloscopio con conector BNC (Male) \\
- & Par de cables Male-Male \\
1 & SAMD \\
\end{longtable}

\subsection{3. Procedimiento}\label{procedimiento}

\begin{enumerate}
\def\labelenumi{\arabic{enumi}.}
\tightlist
\item
  Encender el Generador de Señales y el Osciloscopio.
\item
  Configurar el Generador de Señales para proporcionar una señal
  sinusoidal de 1 KHz de frecuencia, 1.5V de Amplitud y 0V de offset,
  por el canal 1.
\item
  Conectar un extremo del cable BNC en el canal 1 del generador de
  señales y el otro extremo en el canal 1 del osciloscopio.
\end{enumerate}

\begin{figure}[H]

{\centering \includegraphics[width=4.16667in,height=\textheight]{Imagenes/Lab2/Figura1.jpeg}

}

\caption{Fig 1. Conexión del canal del generador de señales}

\end{figure}%

\begin{enumerate}
\def\labelenumi{\arabic{enumi}.}
\setcounter{enumi}{3}
\tightlist
\item
  Mediante los controles de Posición Vertical, Horizontal y Disparo
  ajustar la visualización de la señal sinusoidal.
\end{enumerate}

\begin{figure}[H]

{\centering \includegraphics[width=4.16667in,height=\textheight]{Imagenes/Lab2/Figura2.jpeg}

}

\caption{Fig 2. Ajustar la visualización}

\end{figure}%

\begin{enumerate}
\def\labelenumi{\arabic{enumi}.}
\setcounter{enumi}{4}
\tightlist
\item
  Haciendo uso de los cursores, calcular y mostrar en el osciloscopio
  las medidas de Amplitud y Periodo de la señal.
\end{enumerate}

\begin{figure}[H]

{\centering \includegraphics[width=4.16667in,height=\textheight]{Imagenes/Lab2/Figura3.jpeg}

}

\caption{Fig 3. Mostrar medidas}

\end{figure}%

\begin{enumerate}
\def\labelenumi{\arabic{enumi}.}
\setcounter{enumi}{5}
\item
  Al realizar la verificación de la señal que sale por la sonda del
  generador de señal, se observa que a pesar de configurar un voltaje de
  salida de 1.5V con 0V de offset, la señal es atenuada por la sonda de
  osciloscopio.
\item
  \textbf{Conexión generador de señales y Arduino nano 33 IoT}
\end{enumerate}

\begin{longtable}[]{@{}
  >{\centering\arraybackslash}p{(\columnwidth - 2\tabcolsep) * \real{0.5000}}
  >{\centering\arraybackslash}p{(\columnwidth - 2\tabcolsep) * \real{0.5000}}@{}}
\toprule\noalign{}
\begin{minipage}[b]{\linewidth}\centering
\includegraphics[width=2.60417in,height=\textheight]{Imagenes/Lab2/Figura4.jpeg}
\end{minipage} & \begin{minipage}[b]{\linewidth}\centering
\includegraphics[width=2.60417in,height=\textheight]{Imagenes/Lab2/Figura5.jpeg}
\end{minipage} \\
\midrule\noalign{}
\endhead
\bottomrule\noalign{}
\endlastfoot
\textbf{Fig 4.} Conexión del Arduino sin condensador & \textbf{Fig 5.}
Conexión del Arduino con condensador \\
\end{longtable}

La conexión inicialmente se realizó utilizando dos jumpers, sin embargo,
en las gráficas ploteadas se observaba bastante ruido que impedía
distinguir las señales.

\begin{figure}[H]

{\centering \includegraphics[width=4.16667in,height=\textheight]{Imagenes/Lab2/Figura6.jpeg}

}

\caption{Fig 6. Conexión final}

\end{figure}%

Tras realizar varias pruebas, se observó que de esta manera se mejoraba
la conexión del Arduino Nano 33 IoT con el generador de señales.

\begin{enumerate}
\def\labelenumi{\arabic{enumi}.}
\setcounter{enumi}{7}
\tightlist
\item
  \textbf{Ploteo de señales}
\end{enumerate}

\begin{longtable}[]{@{}
  >{\centering\arraybackslash}p{(\columnwidth - 2\tabcolsep) * \real{0.5000}}
  >{\centering\arraybackslash}p{(\columnwidth - 2\tabcolsep) * \real{0.5000}}@{}}
\toprule\noalign{}
\begin{minipage}[b]{\linewidth}\centering
\includegraphics[width=2.60417in,height=\textheight]{Imagenes/Lab2/Figura7.jpeg}
\end{minipage} & \begin{minipage}[b]{\linewidth}\centering
\includegraphics[width=2.60417in,height=\textheight]{Imagenes/Lab2/Figura8.jpeg}
\end{minipage} \\
\midrule\noalign{}
\endhead
\bottomrule\noalign{}
\endlastfoot
\textbf{Fig 7.} Plot de la señal sin condensador & \textbf{Fig 8.} Plot
de la señal con condensador \\
\end{longtable}

Con las conexiones hechas se obtuvo el ploteo de las señales.
Inicialmente, se realizó sin conectar el condensador y luego con el
condensador. El condensador actúa formando un circuito RC con la
resistencia interna del Arduino, el filtro es del tipo pasa altas.

\subsection{4. Entregable}\label{entregable}

\begin{itemize}
\item ~
  \subsubsection{Plotear al menos 3 señales en Arduino IDE provenientes
  del generador de
  señales.}\label{plotear-al-menos-3-seuxf1ales-en-arduino-ide-provenientes-del-generador-de-seuxf1ales.}
\end{itemize}

\begin{figure}[H]

{\centering \includegraphics[width=4.16667in,height=\textheight]{Imagenes/Lab2/Figura9.jpeg}

}

\caption{Fig 9. Plot señal sinusoidal}

\end{figure}%%
\begin{figure}[H]

{\centering \includegraphics[width=4.16667in,height=\textheight]{Imagenes/Lab2/Figura10.jpeg}

}

\caption{Fig 10. Señal cuadrática}

\end{figure}%%
\begin{figure}[H]

{\centering \includegraphics[width=4.16667in,height=\textheight]{Imagenes/Lab2/Figura11.jpeg}

}

\caption{Fig 11. Plot señal triangular}

\end{figure}%

Se plotearon las 3 señales (Fig 9-11) provenientes del generador de
señales en el Arduino IDE y se obtuvieron gráficas que representan el
tipo de señal generada.

\begin{itemize}
\item ~
  \subsubsection{Comparar las señales graficadas del Arduino IDE con las
  gráficas obtenidas del
  osciloscopio.}\label{comparar-las-seuxf1ales-graficadas-del-arduino-ide-con-las-gruxe1ficas-obtenidas-del-osciloscopio.}
\end{itemize}

\begin{figure}[H]

{\centering \includegraphics[width=4.16667in,height=\textheight]{Imagenes/Lab2/Figura12.jpeg}

}

\caption{Fig 12. Señal sinusoidal del generador de señales}

\end{figure}%%
\begin{figure}[H]

{\centering \includegraphics[width=4.16667in,height=\textheight]{Imagenes/Lab2/Figura13.jpeg}

}

\caption{Fig 13. Señal cuadrática del generador de señales}

\end{figure}%%
\begin{figure}[H]

{\centering \includegraphics[width=4.16667in,height=\textheight]{Imagenes/Lab2/Figura14.jpeg}

}

\caption{Fig 14. Señal triangular del generador de señales}

\end{figure}%

Las señales obtenidas ploteando en el Arduino IDE con las gráficas del
osciloscopio, se observa que las señales en el Arduino IDE presentan un
cierto ruido en las señales comparado a las señales vistas desde el
osciloscopio. La diferencia entre estas señales se puede deber a que se
utilizó un jumper para establecer la conexión. Durante la experiencia se
utilizaron dos jumpers y el ruido disminuyó al conectar una punta de la
sonda directamente al protoboard sin utilizar el jumper. Además, se
pueden generar interferencias por la presencia de otros equipos
electrónicos presentes en la mesa.

\begin{itemize}
\item ~
  \subsubsection{Graficar en Arduino
  cloud.}\label{graficar-en-arduino-cloud.}
\end{itemize}

\begin{figure}[H]

{\centering \includegraphics[width=4.16667in,height=\textheight]{Imagenes/Lab2/Figura15.jpeg}

}

\caption{Fig 15. Código en Arduino cloud}

\end{figure}%%
\begin{figure}[H]

{\centering \includegraphics[width=4.16667in,height=\textheight]{Imagenes/Lab2/Figura16.jpeg}

}

\caption{Fig 16. Gráfica sinusoidal de frecuencia 1 Hz, frecuencia de
muestreo 20 Hz}

\end{figure}%%
\begin{figure}[H]

{\centering \includegraphics[width=4.16667in,height=\textheight]{Imagenes/Lab2/Figura17.jpeg}

}

\caption{Fig 17. Gráfica cuadrática de frecuencia 1 Hz, frecuencia de
muestreo 20 Hz y condensador (al inicio)}

\end{figure}%%
\begin{figure}[H]

{\centering \includegraphics[width=4.16667in,height=\textheight]{Imagenes/Lab2/Figura18.jpeg}

}

\caption{Fig 18. Gráfica cuadrática de frecuencia 1 Hz, frecuencia de
muestreo 20 Hz y condensador (ya con tiempo)}

\end{figure}%%
\begin{figure}[H]

{\centering \includegraphics[width=4.16667in,height=\textheight]{Imagenes/Lab2/Figura19.jpeg}

}

\caption{Fig 19. Gráfica sinusoidal de frecuencia 1 Hz, frecuencia de
muestreo 10 Hz}

\end{figure}%%
\begin{figure}[H]

{\centering \includegraphics[width=4.16667in,height=\textheight]{Imagenes/Lab2/Figura20.jpeg}

}

\caption{Fig 20. Gráfica sinusoidal de frecuencia 1 Hz, frecuencia de
muestreo 2 Hz}

\end{figure}%%
\begin{figure}[H]

{\centering \includegraphics[width=4.16667in,height=\textheight]{Imagenes/Lab2/Figura21.jpeg}

}

\caption{Fig 21. Gráfica sinusoidal de frecuencia 500 Hz, frecuencia de
muestreo 20 Hz con capacitor}

\end{figure}%%
\begin{figure}[H]

{\centering \includegraphics[width=4.16667in,height=\textheight]{Imagenes/Lab2/Figura22.jpeg}

}

\caption{Fig 22. Gráfica sinusoidal de frecuencia 10 Hz, frecuencia de
muestreo 20 Hz con capacitor}

\end{figure}%

\subsubsection{Observaciones e
Interpretaciones:}\label{observaciones-e-interpretaciones}

\begin{itemize}
\item
  Condensador en el circuito:

  El condensador en el circuito estaría actuando como un filtro pasa
  alta, ya que permite el paso de frecuencias más altas mientras atenúa
  las frecuencias más bajas. Así afecta la forma en que se detecta y
  registra la señal en el Arduino por el generador. En el laboratorio
  contamos con un condensador de valores muy altos que no facilitaron la
  visualización para ver la frecuencia de corte, ya que era de 16V y 470
  uF. Necesitaríamos la resistencia interna y se calcularía: frecuencia
  de corte = 1/(2 x pi x Resistencia x Capacitancia)
\item
  Sin con el condensador:

  La señal captada por el Arduino tendrá no presentará ninguna
  atenuación extra en las frecuencias bajas, entonces se conserva la
  señal original del generador de ondas.
\item
  Filtro pasa alta frecuencia:

  Significa que permite el paso de frecuencias altas y atenúa las
  frecuencias más bajas. Esto puede ser útil para eliminar componentes
  de baja frecuencia no deseados o ruido de la señal captada por el
  Arduino.
\item
  Frecuencia:

  Al variar la frecuencia de la señal de entrada nos dimos cuenta que
  esta afectaa la forma en que se captura y registra en el Arduino para
  la tomas de datos. Frecuencias más altas pueden requerir una
  frecuencia de muestreo más alta para capturar con precisión la forma
  de onda completa. Es por eso que requerimos un mayor valor que el
  doble de la frecuencia máxima para tener una mejor visualización de
  ploteo.
\item
  Atenuación:

  La atenuación de la señal ocurre debido a la resistencia interna del
  generador de ondas, la resistencia de los cables y otras resistencias
  en el circuito. La presencia del condensador también influye en la
  atenuación de la señal.
\item
  Ruido del cable:

  El ruido del cable se visualiza por la interferencia electromagnética
  ambiental o a la capacitancia entre los cables de señal y tierra. Se
  manifiesta como fluctuaciones aleatorias en la señal capturada por el
  Arduino.
\item
  Cables BNC:

  La boquilla, la calidad y la longitud de los cables BNC afectan la
  integridad de la señal capturada por el Arduino. Así inferimos que sin
  boquilla y con cables de alta calidad se minimiza la atenuación y la
  interferencia no deseada.
\item
  Comparar con el osciloscopio:

  Es importante comparar la señal capturada por el Arduino con la señal
  observada en un osciloscopio para verificar la precisión y la
  integridad de la captura de datos. El osciloscopio en el laboratorio
  nos sirvió de guía para proporcionar una representación visual más
  precisa de la forma de onda y puede revelar detalles sutiles que
  pueden perderse en la lectura digital del Arduino, por ejemplo con los
  datos del voltaje.
\end{itemize}




\end{document}
